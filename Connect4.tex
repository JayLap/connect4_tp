\documentclass[12pt]{article}
\usepackage{graphicx}
\usepackage[utf8]{inputenc}
\usepackage[a4paper,left=2.5cm,right=2.5cm,top=2.5cm,bottom=2.5cm]{geometry}
\usepackage[frenchb]{babel}
\usepackage{setspace}
\onehalfspacing
\begin{document}

\begin{titlepage}
	\begin{minipage}{0.5\textwidth}
		\begin{flushleft}
			\includegraphics[scale=0.1]{logo.png} \\
			\textbf{Département des sciences et génie}
		\end{flushleft}
	\end{minipage}
	\vspace{150px}
	\begin{center} \large
		\textbf{Connect Four : TP2} \\
		IFT-2003 \\
		Intelligence artificielle I \\
		\vspace{150px}
		\textbf{Destinataire} \\
		Laurence Capus
	\end{center}
	\vfill
	\rule{\linewidth}{.5pt}
	\newline
	\textbf{Catherine Asselin, 111 128 110} \\
	\textbf{Jérémie Lapointe, 111 008 937} \\
	\textbf{Patrick Voyer, 111 152 697}
	%\textbf{Petru Lungu, } \\
\end{titlepage}

\newpage
\setcounter{page}{1}
\pagenumbering{arabic}

\section*{Introduction}
Le but de ce travail est de concevoir un jeu en utilisant des techniques d'intelligence artificielle, et ce, afin de mieux comprendre les différents concepts de l'intelligence artificielle. Notre équipe a choisi de concevoir le jeu de \textit{Puissance 4}, un jeu simple qui s'apparente au tic tac toe. Nous voulions créer un jeu qui se joue en duel avec l'ordinateur en utilisant un interface graphique de base.
%travail réalisé

\section*{Présentation du jeu choisi}
Le jeu que nous avons choisi de concevoir est le célèbre jeu \textit{Connect Four} ou \textit{Puissance 4}. Ce jeu est composé d'un plateau à trous de 7 colonnes et de 6 rangées. Il est possible d'insérer des jetons par le haut du plateau dans chacune des colonnes. Le jeton inséré se retrouve alors au bas de la colonne. \\

Le but du jeu est d'être le premier joueur à aligner 4 jetons de sa couleur sur n'importe quelle ligne ou diagonale du plateau. Chacun leur tour, les deux joueurs doivent mettre un jeton dans l'une des 7 colonnes du plateau, jusqu'à ce qu'il y est un vainqueur. Si toutes les rangées et colonnes sont pleines, et ce, sans qu'aucun des joueurs n'est aligné 4 jetons, la partie est déclarée nulle.

\section*{Modélisation du problème}

\subsection*{État initial}
L'état initial du jeux Puissance 4 sera représenté en utilisant des listes. Il y aura donc 7 listes qui permettront de représenter les différentes colonnes du jeu. Ces listes contiendront chacune 6 éléments pour représenter les emplacements ou peuvent être situé des jetons. De cette façon, nous obtenons un état initial qui a 7 colonnes et 6 lignes. Lorsque l’on initialise le jeu, une valeur indiquant que la case est vide est inséré dans chacun des 42 emplacements disponibles sur le jeu. L’état initiale contient aussi une valeur indiquant qu’elle joueur commence la partie.

\subsection*{États finaux}
Les états finaux du jeu représentent toutes les combinaisons possibles pour qu’un joueur remporte la partie. Pour gagner une partie, un joueur doit avoir quatre jetons de sa couleur alignée sur une colonne, une rangée ou une diagonale. Les états finaux doivent donc permettre de vérifier à chaque tour joué si les jetons d’un joueur respectent une des 69 combinaisons possibles. Plus spécifiquement, une fonction permettra de prendre la couleur du joueur qui vient de poser une pièce et parcourir toutes ces combinaisons pour déterminer si ce joueur a gagné. Si une des conditions est respectée, ce joueur est identifié comme gagnant et la partie se finit. Si aucune des combinaisons n’est identifiée, le joueur n’a pas gagné et c’est au second de joueur de jouer. 

\subsection*{Mouvements autorisés}
Le seul mouvement du jeux est l'insertion d'un jeton dans l'une des colonnes du jeux. Il existe seulement deux restrictions pour l'insertion d'un jeton. Tout d'abord, il faut que le joueur qui procède à l'insertion soit le joueur courant. La deuxième restriction est que la colonne dans laquelle le joueur veut insérer son jeton ne doit pas être pleine. C'est-à-dire qu'il ne faut pas qu'il y est déja 6 jetons dans la colonne. \\

Plus spécifiquements, la première condition sera toujours géré par la boucle du jeux  et fera en sorte que le joueur qui choisi sera toujours celui qui doit réellement jouer. Pour la deuxième condition, à chaque fois qu'un joueur choisira une colonne, il faudra vérifier si il reste une valeur indiquant une case vide dans la liste représentant cette colonne. Si cette valeur existe, le joeur pourra procéder à l'insertion, sinon, le joueur devra choisir une nouvelle colonne. 

\section*{Résultats et discussion}

\section*{Conclusion}
%résumé du travail accompli et ce que vous auriez pu ou aimé faire de plus

\end{document}
